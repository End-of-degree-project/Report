\newpage
\section{Extensions}

Both possible ways for extending Wave have been explored in this work. Gadgets have been implemented with the Wave Gadgets API, and robots with the Java version of the robots API.

\begin{figure}[h]
  \center
    \includegraphics[keepaspectratio, scale=0.75]{Media/Diagrams/Wave/Structure.png}
  \caption{Wave Structure}
  \label{fig:wave_structure}
\end{figure}

To understand where these extensions live, it is necessary to understand how wave is structured. In Figure \ref{fig:wave_structure} you can see how Wave documents are structured. The outermost layer is the wave, a thread or conversation, the whole picture of a conversation. Users are invited to the wave, letting them participate in any of the inner wavelets, when they then become participants. As of today, in current versions of Wave In A Box and Kune, there is only one wavelet inside every wave, but the Wave Protocol allows for multiple wavelets. Blips are individual messages created by a participant, and edited by participants later. Blips have hierarchical structure and can be nested inside other blips. Blips also contain a document in XML-like format, which is the text itself plus all the variations that Wave supports.\\[.2cm]
Gadgets are inserted inside a document, together with the content itself. They are placed inside a blip, but belong to the wave and therefore have access to the wave's content, wich is given access by the Gadgets API. But that access is very limited, almost only to access what is called the wave's state, a key-value dictionary that keeps track of every state change that occurs inside the wave.\\[.2cm]
There is actually two differentiated kind of states in a wave:
\begin{itemize}
  \item Private State: Stores information that can only be accessed and modified from the participant that created it. Useful to keep stored private information that is not intended to be shared.
  \item Shared State: Stores the global state of every gadget in this wave, representing the whole picture. Useful for communicating with the other participants and collaborating or comunicating in the gagtet.
\end{itemize}
It is important to know that the state belongs to the wave, so gadgets will have access to the other gadgets state, even different instances of the same gadgets. It is then good practice to adopt a system similar to namespaces in programming languages, appending an identifier of the gadget before the key in a way we are not altering or reading an unintended state.\\[.2cm]
Each actual instance of the gadget in a participant's browser is local, variables are not shared between different participants. When a user triggers a state change in his local instance of the gadget, the only thing that can be communicated outside of the browser is a delta, that is a change in the key-value pairs that represent the state. This delta is then communicated to the local instance of the gadget in every client's browser. That behaviour can be seen represented in figure \ref{fig:wave_state}.
\begin{figure}[h]
  \center
    \includegraphics[keepaspectratio, scale=0.6]{Media/Diagrams/Wave/StateSequence.png}
  \caption{Wave state transmission}
  \label{fig:wave_state}
\end{figure}


\begin{center}
------------------------------------------------------------------------------------------\\
\end{center}

Wave gadgets live inside a wave, and therefore are able to interact with it. Among other things, they can access the current state of the wave. There is two different kinds of state:

\begin{itemize}
  \item Private State\\
        Stores information that can only be accessed from the participant that modified it.
  \item Shared State\\
        Stores the global state of every gadget in this wave.
\end{itemize}

The state is actually an entity containing key-value pairs of strings. The state can be modified at any time, and the changes made to it will be communicated to the rest of the participants, so they can react accordingly to this change. This is represented in the next figure:

\anotacion{TODO: Explain the diagrams a little}

\begin{center}
\includegraphics[keepaspectratio, scale=0.6]{Media/Diagrams/Wave/StateSequence.png}
\end{center}


The Gadgets API requires you to create a class hierarchy to be able to succesfully compile. In a hypothetical gadget named ``Gadget'' it would be as follows:

\begin{center}
\includegraphics[keepaspectratio, scale=0.5]{Media/Diagrams/Gadget/Gadget.png}
\end{center}

For every gadget project it has also been implemented a tester and a deployer.

The intention of the tester is to take advantage of the ``Development Mode'' in GWT that allows you to run the code locally without deploying it to a web server. The class diagram for every tester project has followed the same pattern:

\begin{center}
\includegraphics[keepaspectratio, scale=0.6]{Media/Diagrams/Gadget/Tester.png}
\end{center}

The project is then run as a Web Application with Google's App Engine.

The deployer also follows a similar pattern:

\begin{center}
\includegraphics[keepaspectratio, scale=0.5]{Media/Diagrams/Gadget/Deployer.png}
\end{center}

The difference between this two structures are the following: First, a Wave Mock is no longer needed, as our gadget will be in an environment with a real Wave behaviour, so the Wave class is bound from the GinModule. Also, the entry point now also extends the Gadget class, a class given by the Gadgets API which some behaviour needed for the Gadget to work in an actual environment.

\thispagestyle{sectioned}
\chapter{CCWave: A Wave Gadget}
\label{subsec:cc_intro}
%\addcontentsline{toc}{chapter}{\numberline{Creative Commons Gadget}}
As it can be seen in the Design Guidelines of the P2Pvalue project, in the Legal Regime section \cite{ref:p2pvalue}, they identify software licenses as a key feature: ``We should secure that none of the content produced / uploaded on the platform is infringing: Draft specific terms of use (if we use the Wikipedia model, rather than the Kune model)  or include a mention during sign-up process for contributors to accept to \textbf{release their work/data under a particular license}'' and state the importance of keeping information free ``A large number of respondents described that the \textbf{commercialization of the commons by third parties would reduce their motivation to contribute}. Provide guidance mechanism to help users choose between different licensing schemes based on the interests of the commoners involved in the project''.\\[.2cm]
The default license for any creation of any kind is the restrictive copyright. Copyright tries to ensure that all the rights remain to the original author of the content, but sometimes it can be advantageous to let people freely or semi-freely use, distribute or modify the content \cite{ref:oss_why}. \textbf{One of the most popular of these licenses}, with over 400 million \cite{ref:the_power_of_open} licensed works, are the Creative Commons set of licenses. These are not adequate for licensing software, but in the context of Wave the content is usually creative text work, for which Creative Commons suits perfectly.\\[.2cm]
Thanks to the development of this Gadget, it has been documented in the P2Pvalue wiki's article regarding Gadget Development the need for an OpenSocial server in order to run Gadgets \cite{ref:gadget_development}.

\section{State of the Art}
\label{subsec:cc_soa}
Wave does not have any way of publishing your content under any specific license. Kune has the option for publishing Waves in your personal space under one of the existing Creative Commons Licenses. Outside of the personal space, Kune preserves Wave's aspect and removes the space where the license is in the personal space, so no license is imposed to other participants in a wave.\\[.2cm]
Outside the world of wave, there's similar alternatives to the one trying to be implemented here. There is for example Creative Commons Configurator for Wordpress, which lets you set a license for your Wordpress posts. Creative commons also has a license chooser \cite{ref:cc_chooser} in their own webpage, that selects a license based on questions asked to learn the needs of the user.\\[.2cm]
There is a limited amount of Creative Commons licenses, all of them requiring attribution of the original creator. There is some concepts which can be combined to build the different licenses:
\begin{itemize}
  \item Attribution: The used of this work has to be attributed to the original author and distributed with this attribution. If the BY is present, attribution is required.
  \item Derivation: Derivative works are those based on a previous work and modified to make a new work. If the ND is present, no derived works are allowed.
  \item Commercial: A commercial work is that which seeks a profit directly or indirectly. If the NC is present, no commercial use of this work is allowed.
  \item Share Alike: Share the work the same way it was previously shared. If SA is present, usage of this work is forced to be shared alike.
\end{itemize}
Not all combinations of this concepts are possible, Table \ref{fig:cc_licenses} shows the existing licenses and their denominations.
\begin{table}[H]
  \begin{center}
    \begin{tabular}{ | l | c |}
      \hline
      CC-BY & Attribution\\
      CC-BY-SA & Attribution + Share Alike\\
      CC-BY-NC & Attribution + Non Commercial\\
      CC-BY-NC-SA & Attribution + Non Commercial + Share Alike\\
      CC-BY-ND & Attribution + No Derivatives\\
      CC-BY-NC-ND & Attribution + Non Commercial + No derivatives\\
      \hline
    \end{tabular}
  \end{center}
  \caption{Available Creative Commons licenses}
  \label{fig:cc_licenses}
\end{table}
\section{Results}
This extension allows participants to \textbf{set a Creative Commons license} to a specific blip, by inserting a gadget in the content and answering questions to reach the adequate license that meets the requirements. The Figure \ref{fig:cc_gadget} shows the final result of this extension.\\[.2cm]
Relating to the gadget structure described in Figure \ref{fig:gadget_classes}, this Gadget's Composite is the class CCGadgetMainPanel, the Messages relates to CCGadgetMessages and the GinModule would be represented by the class CCGadgetGinModule.
\begin{figure}[H]
  \center
    \includegraphics[keepaspectratio, scale=0.7]{Media/Captures/Extensions/CCGadget.png}
  \caption{CCWave}
  \label{fig:cc_gadget}
\end{figure}
This gadgets keeps a variable for each one of the possible concepts for a unique Creative Commons License, such as ``Commercial use'', and \textbf{asks a question to the user} to determine if the restriction has to happen or not, in order to figure out a valid license. The state stores the state of all the answers to the questions that can be asked, and the answers can be yes, no, or unanswered. When a question is answered a state delta is sent to the rest of the participants, and the image of the license is update to the best match from all the possible licenses. When all the answers are answered, only the image of the license remains linking to the whole chosen license.\\[.2cm]
\section{Conclusions and Future Work}
This extension is a Gadget that can be inserted in a blip to let the user select a Creative Commons license for the content of the blip.\\[.2cm]
The main problem with this extension is the \textbf{limited amount of licenses}, only Creative Commons. Even though they are widely used, they might not cater to the tastes of everyone, or fit every possible position. There is also no way to license anything other than a blip, such as a blip and all its children or a whole wave.\\[.2cm]
Those are limitations that can be solved starting with this extension and adding the possibility for more or different licenses. Solving the problem of licenses only being in a blip could not be done with Gadgets as they are by definition inside a blip, it would be compulsory to find a different solution.
\newpage

\section{Decision Maker Gadget}
When talking in a group of people it is sometimes necessary to take a decision about a specific aspect. The easiest way to achieve this is by talking, and Wave with real time communication and editing common documents makes this easy. But this way it might be difficult to easily differentiate all the different options, see who agrees with each one of the options, or quickly decide one. This extension allows every participant to choose an option or add new options to the decision. Participants are also able to quickly see who voted each option along with their user avatar images. This features accompanied by Wave's style of communication make consensus and decision making easier.

\label{subsec:decision_soa}
\subsection{State of the Art}
Because of the nature of Wave, decision making is a very important and basic, so there has been several gadgets made regarding that. It also easily explores all the basic possibilities that the Gadgets API provides. Figures \ref{fig:poll} and \ref{fig:consensuall} show two of them: The first one is Poll by Eric Williams, focused on showing statistics about the results, maybe useful when there is a high amount of votes. The second one is Consensuall by Antonio Tenorio, focused on sharing the personal opinion of each voter about an issue and reaching consensus on it.\\[.2cm]
\begin{figure}[H]
  \center
    \includegraphics[keepaspectratio, scale=0.7]{Media/Captures/Extensions/DecisionGadgets/other.png}
  \caption{Poll}
  \label{fig:poll}
\end{figure}
\begin{figure}[h]
  \center    
    \includegraphics[keepaspectratio, scale=0.7]{Media/Captures/Extensions/DecisionGadgets/consensuall.png}
  \caption{Consensuall}
  \label{fig:consensuall}
\end{figure}
Decision Maker Gadget falls in between both of them by letting people add their own answers, see a representation of the votes and who personally voted for each.

\subsection{Results}
Decision Maker Gadget is a gadget that can be inserted anywhere in a blip. A title is chosen for the issue at hand, and then different options answering that title can be inserted by anyone who sees the gadget, not only the creator. One of the decisions can be instantly chosen by clicking on it.
\begin{figure}[H]
  \center
    \includegraphics[keepaspectratio, scale=0.4]{Media/Captures/Extensions/DecisionMakerGadget.png}
  \caption{Decision Maker Gadget}
  \label{fig:decision_maker_gadget}
\end{figure}
When there have been decisions chosen by different participants, hovering the mouse over the number of votes on the right will show the names and pictures of the participants that voted for it, as represented in Figure \ref{fig:decision_maker_votes}. Names and pictures are taken from their Wave profile.
\begin{figure}[H]
  \center
    \includegraphics[keepaspectratio, scale=0.4]{Media/Captures/Extensions/DecisionMakerGadget_votes.png}
  \caption{Decision Maker Voters}
  \label{fig:decision_maker_votes}
\end{figure}
This gadget also relates to the structure shown in Figure \ref{fig:gadget_classes}. The GinModule is the DecisionMakerGinModule, the Composite is the DecisionMakerMainPanel, and the Messages class is represented by the class DecisionMakerMessages. This gadget is slightly more complex than the Creative Commons one, so the class diagram with its specifics is shown in Figure \ref{fig:decision_maker_diagram}.
\begin{figure}[H]
  \center
    \includegraphics[keepaspectratio, scale=0.5]{Media/Diagrams/Gadget/DecisionMaker.png}
  \caption{UML Class Diagram, Decision Maker}
  \label{fig:decision_maker_diagram}
\end{figure}
The interface DecisionManager is extended. It is meant to represent all that can be done with the decisions: select one, add a new decision, and get the total votes of one specific decision. The DecisionMakerMainPanel implements it, and is also the container for all the decisions. A decision represents one of the options that can be chosen, and generates the Popup showing who voted for it. A StateManager is also used to handle everything related to the Wave's state.\\[.2cm]
There are three different types of entries in the Wave's state:
\begin{itemize}
  \item Count of votes: There is one entry of this kind for each decision. To identify which decision this entry is for, the title of the decision is stored as a key. Therefore, the decision titles have to be unique. The value of this entry is the amount of votes that decision has, making it quick to retrieve the amount and show the vote count.
  \item Voters: Again, one entry for each decision and the title of the decision as an unique identifier. The value is a list of all the people that voted for that decision. As the state is only able to store a string, the list is like \verb+|user1@kune.cc|user2@kune.cc|user3@kune.cc|+. This entry is used to know if a particular user has voted for a decision, and to fill the information inside the votes popup.
  \item Title: There is one single entry for each gadget. The value of this entry is the name of the decision to be taken. Used to fill the title after it has been set.
\end{itemize}
\subsection{Conclussions and Future Work}
There are already several alternatives for decision making, each one with their unique way of doing things, and there is certainly already one suitable for every need, and Decision Maker doesn't innovate in almost any aspect. This alternative has limitations though: There is no option for multi-choice answers, no way of editing the title after being created, and the list of votes can get a little uncomfortable to see after lots of votes have been made. Multi-choice answers would put this gadget closer to a consensus tool, and being able to block options would make the divergence among opinions more visible.

\thispagestyle{sectioned}
\chapter{Video Conference Gadget}
%\addcontentsline{toc}{chapter}{\numberline{Video Conference Gadget}}
Wave is meant for text communication. Text has the advantage that it can be stored easily, searched later and edited simultaneously, but lacks the naturality of human-to-human interaction. To get as close as possible to it we need voice and video. Then collaborating on text can be made more efficiently. This gadget makes it possible to \textbf{speak to up to 16 other users and see them}.

\label{subsec:video_soa}
\section{State of the Art}
There is no alternative to video or audio communication integrated on Wave. There is other alternatives though outside of it, some of them shown in Figure \ref{fig:skype_hangouts}.
\begin{figure}[h]
  \center
    \includegraphics[keepaspectratio, scale=0.6]{Media/Captures/Soa/skype_hangouts.png}
  \caption{Hangouts and Skype}
  \label{fig:skype_hangouts}
\end{figure}
They focus on video, even hiding text communication to leave more room to the video. The Video Conference Gadget can be mixed with text above and below, leaving the video communication as an addition and not the main point. Both video communication tools shown (Google's Hangouts and Microsoft's Skype) require you to have a specific user account to use their services, while this gadget lets you join the \textbf{video communication from within Wave}, but also from an external link without giving any kind of personal information.

\section{Results}
To make this gadget it has been essential the use of a pre-existing service called \textbf{appear.in} \cite{ref:appearin}. They provide the whole video and audio communication based on WebRTC \cite{ref:webrtc}, and also facilitate an easy way to use their service in an external website. It is a JavaScript component that can be put inside an iframe, and it will take care of almost everything. Figure \ref{fig:video_gadget} shows the final result of this integration.
\begin{figure}[H]
  \center
    \includegraphics[keepaspectratio, scale=0.8]{Media/Captures/Extensions/VideoGadget/RoomSelection.png}
  \caption{Room Selection Screen}
  \label{fig:video_gadget_room}
\end{figure}
When a user inserts the gadget, he will be prompted with the room selection screen shown in Figure \ref{fig:video_gadget_room}, letting him choose the identifier of the room people will meet in. The same room can be visited from different waves, and also directly from appear.in, as room names can not be duplicated.
\begin{figure}[H]
  \center
    \includegraphics[keepaspectratio, scale=0.45]{Media/Captures/Extensions/VideoGadget.png}
  \caption{Video Conference Gadget}
  \label{fig:video_gadget}
\end{figure}
It was shown in Figure \ref{fig:gadget_classes} the basic structure of a gadget. The composite is represented by the VideoGadgetMainPanel, the Messages are the class VideoGadgetMessages, and the GinModule is realized by VideoGadgetGinModule. Outside from that, the structure is relatively simple.\\[.2cm]
As appar.in is a JavaScript service, multiple calls to \textbf{native JavaScript} have been made inside this gadget. To do that, you have to write the following pattern (Example for a private function returning a void type):
\begin{verbatim}
private native void function() /*-{
  Native JavaScript code goes here
}-*/;
\end{verbatim}
The service appear.in uses the name of the room as a unique identifier, so the gadget asks the user for a name before entering the room. The gadget also makes use of appear.in's \verb|?lite| feature, that simplifies the user interface, leaving more space for the images of the video. The \textbf{camera is accessed through the browser}, so no additional software has to be installed.\\[.2cm]
The Wave state in this gadget is really simple, the only thing stored in it is the name of the room to enter it directly if it has already been set.\\[.2cm]
\begin{figure}[h]
  \center
    \includegraphics[keepaspectratio, scale=0.6]{Media/Diagrams/Gadget/VideoSequence.png}
  \caption{UML Sequence Diagram, Video Gadget}
  \label{fig:video_gadget_sequence}
\end{figure}
Figure \ref{fig:video_gadget_sequence} shows how the gadget reacts to requests. Once the gadget is inserted, the room selection screen will be shown, allowing any participant to select the name of the room the participants will meet in. When a room is selected, the participants automatically enter the room. If another participant joins the wave once the room has been selected, he will be served the video conference directly.
\section{Conclusions and Future Work}
The fact that this extension is completely \textbf{dependant on an external closed-source service} is a big limitation. The service could stop working at any time, technical improvements on video or audio communication can not be made, limitations like the maximum of 16 online users can not be avoided, the way to use it could change making it necessary to update the gadget, and other changes may arise.
\newpage

\thispagestyle{sectioned}
\section{Colorizer Robot}
Again, text communications gets in the way of collaboration in some aspects. When working between several people it is sometimes necessary to talk about some aspect of the work they disagree in. In plain text, when there are more than two collaborators, there is no \textbf{way to know who edited what}, so those kind of issues can't be addressed personally. A robot is needed in order to know when and what changes are being made.

\label{subsec:color_soa}
\subsection{State of the Art}
In Wave it is possible to see who has participated in a blip as shown in Figure \ref{fig:participants}, but it is \textbf{not possible to see exactly what that participant has modified} what part of the document.
\begin{figure}[H]
  \center
    \includegraphics[keepaspectratio, scale=0.7]{Media/Captures/Wave/Participants.png}
  \caption{Blip Participants in Apache Wave}
  \label{fig:participants}
\end{figure}
The other thing Wave does to try to keep people informed on when changes happen, is to highlight the changes that just happened and write the author's name next to it, as shown in Figure \ref{fig:participants2}. The problem is it is not permanent, so you only realize of the change if you were already looking at the content being changed.
\begin{figure}[H]
  \center
    \includegraphics[keepaspectratio, scale=0.7]{Media/Captures/Wave/Participants2.png}
  \caption{Change Highlighting in Apache Wave}
  \label{fig:participants2}
\end{figure}
Also, this extension is heavily inspired in Pads, services like PiratePad, Etherpad, TitanPad and many others, that offer an online collaboration tool that allows to concurrently write plain-text documents. They show a specific color for each participant to quickly see who edited what. TitanPad in Figure \ref{fig:titanpad}, even though not the only, has the capability to show a timeline and revisit past states of the Pad, so no information is lost even after being modified.
\begin{figure}[h]
  \center
    \includegraphics[keepaspectratio, scale=0.4]{Media/Captures/Soa/TitanPad.png}
  \caption{TitanPad}
  \label{fig:titanpad}
\end{figure}

\subsection{Results}
This extension, in the shape of a robot, goes around that problem by \textbf{assigning a color to each participant}, and painting the background of the text that participant edits. The result can be seen in Figure \ref{fig:colorizer_editions}. It also keeps track of who has each color and puts it in a blip under the main blip of the wave, as seen in Figure \ref{fig:colorizer_editors}. It is also possible to get around the colorizing of any given blip by starting it with \verb|@Robot clear annotations|, being ``Robot'' the actual name of the robot that it was registered with.\\[.2cm]
\begin{figure}[H]
  \center
    \includegraphics[keepaspectratio, scale=0.8]{Media/Captures/Extensions/Colorizer/ColorizerEditions.png}
  \caption{Colorizer Robot Colors}
  \label{fig:colorizer_editions}
\end{figure}
\begin{figure}[h]
  \center
    \includegraphics[keepaspectratio, scale=0.7]{Media/Captures/Extensions/Colorizer/ColorizerEditors.png}
  \caption{Colorizer Robot Tracking Participants}
  \label{fig:colorizer_editors}
\end{figure}
The way to make the background of the text be of a specific color is by changing the annotations of the document. Annotations in Wave are tags affecting a range of text and altering its properties, but they don't affect the text itself. Annotations are also able to be transmitted through the Federation Protocol. Every annotation is defined by a name (Specifying its purpose), a range (Determining the range of characters of text it affects), and a value (The value should that annotation take on that range). There are annotations for text size, links, language, among other. For this robot specifically the annotation \textbf{style/backgroundColor} is the one being set in the changed text.\\[.2cm]
The way to know when the changes happen is by subscribing th the \textbf{DocumentChanged event} explained before. This event will be triggered anytime anyone modifies the text.\\[.2cm]
\begin{figure}[H]
  \center
    \includegraphics[keepaspectratio, scale=0.5]{Media/Diagrams/Robot/Colorizer.png}
  \caption{UML Class Diagram, Colorizer Robot}
  \label{fig:colorizer_diagram}
\end{figure}
That only leaves one task left: The DocumentChanged event tells us the document has changed, but not \textbf{what part of the document changed} or how it changed, so it is up to us to extract that information. To achieve it Google's google-diff-match-patch, a Java library that is able to extract the difference between two chunks of plain text. Every time the document is changed, the difference with the last version is calculated, and the new content is attributed to the participant that changed it.\\[.2cm]
Figure \ref{fig:colorizer_diagram} represents the class structure of the Colorizer Robot. The HttpServlet is the responsible of receiving the communication from the Wave Protocol. The ColorizerRobot sets up the OAuth authentication with the token received when registering the robot. Also, thanks to the AbstractGadget class it is registered to receive all the events. The GuiceServletConfig binds the necessary dependencies together.
\begin{figure}[H]
  \center
    \includegraphics[keepaspectratio, scale=0.65]{Media/Diagrams/Robot/RobotActivity.png}
  \caption{UML Activity Diagram, Colorizer Robot}
  \label{fig:colorizer_activity_diagram}
\end{figure}
Figure \ref{fig:colorizer_activity_diagram} shows the behaviour of this robot in time, disregarding the aspects of the robot registration. Once the server is started, it is ready to be used as a robot and \textbf{added as a participant to the wave}. When it is added, the first thing done is creating the blip for tracking the participants that have acted. When the blip is created, it will start listening for Document Changed events. When each of those events is received, the robot will check if the participant that generated the event is already being tracked, and add it otherwise. The Document Changed event is triggered also when the document is changed by an equivalent one, so the difference of both documents has to be calculated. Annotations are updated according to the participants and changes. Afterwards, it goes back to waiting to another Document Changed event.

\subsection{Conclusions and Future Work}
This extension, as said before, uses an API for diff calculation that works with plain text. That means changes on annotations are not detected. Also, diff detection can be problematic in some specific cases. The difference between two documents is always computed correctly, but as the library does not have information on where the text changed, and it is not possible to extract that information from within the robots API, it will sometimes attribute a change to a wrong position of the text. For example: We have the string ``abc'', and a participant decides to add a new ``b'' to the left of the already existing ``b''. The end result would be ``abbc'', where the first apparition of the letter will be the change from the last version of the text. But because of how the diff calculation is made, the library will come up with the result that the difference between those two texts is the ``b'' on the right and the robot will put the background color in the wrong character. This small example can be extrapolated to more complex and problematic cases.\\[.2cm]
Wave, thanks to the federation protocol and the deltas, is able to keep track of all the history of the modifications in the document. Also because of deltas, changes can be attributed to a participant with little overhead, as opposed to calculating the difference every time a change is made. With past states it would be possible to reliably go back in history and review what happened. \textbf{Implementing this feature at a Wave protocol level} would definitely be advantageous.


\anotacion{Talk about how gadgets and robots can be run (tomcat, shindig, jetty)}

\anotacion{Talk about documentation}
