\thispagestyle{sectioned}
\chapter{CCWave: A Wave Gadget}
\label{subsec:cc_intro}
%\addcontentsline{toc}{chapter}{\numberline{Creative Commons Gadget}}
As it can be seen in the Design Guidelines of the P2Pvalue project, in the Legal Regime section \cite{ref:p2pvalue}, they identify software licenses as a key feature: ``We should secure that none of the content produced / uploaded on the platform is infringing: Draft specific terms of use (if we use the Wikipedia model, rather than the Kune model)  or include a mention during sign-up process for contributors to accept to \textbf{release their work/data under a particular license}'' and state the importance of keeping information free ``A large number of respondents described that the \textbf{commercialization of the commons by third parties would reduce their motivation to contribute}. Provide guidance mechanism to help users choose between different licensing schemes based on the interests of the commoners involved in the project''.\\[.2cm]
The default license for any creation of any kind is the restrictive copyright. Copyright tries to ensure that all the rights remain to the original author of the content, but sometimes it can be advantageous to let people freely or semi-freely use, distribute or modify the content \cite{ref:oss_why}. \textbf{One of the most popular of these licenses}, with over 400 million \cite{ref:the_power_of_open} licensed works, are the Creative Commons set of licenses. These are not adequate for licensing software, but in the context of Wave the content is usually creative text work, for which Creative Commons suits perfectly.\\[.2cm]
Thanks to the development of this Gadget, it has been documented in the P2Pvalue wiki's article regarding Gadget Development the need for an OpenSocial server in order to run Gadgets \cite{ref:gadget_development}.

\section{State of the Art}
\label{subsec:cc_soa}
Wave does not have any way of publishing your content under any specific license. Kune has the option for publishing Waves in your personal space under one of the existing Creative Commons Licenses. Outside of the personal space, Kune preserves Wave's aspect and removes the space where the license is in the personal space, so no license is imposed to other participants in a wave.\\[.2cm]
Outside the world of wave, there's similar alternatives to the one trying to be implemented here. There is for example Creative Commons Configurator for Wordpress, which lets you set a license for your Wordpress posts. Creative commons also has a license chooser \cite{ref:cc_chooser} in their own webpage, that selects a license based on questions asked to learn the needs of the user.\\[.2cm]
There is a limited amount of Creative Commons licenses, all of them requiring attribution of the original creator. There is some concepts which can be combined to build the different licenses:
\begin{itemize}
  \item Attribution: The used of this work has to be attributed to the original author and distributed with this attribution. If the BY is present, attribution is required.
  \item Derivation: Derivative works are those based on a previous work and modified to make a new work. If the ND is present, no derived works are allowed.
  \item Commercial: A commercial work is that which seeks a profit directly or indirectly. If the NC is present, no commercial use of this work is allowed.
  \item Share Alike: Share the work the same way it was previously shared. If SA is present, usage of this work is forced to be shared alike.
\end{itemize}
Not all combinations of this concepts are possible, Table \ref{fig:cc_licenses} shows the existing licenses and their denominations.
\begin{table}[H]
  \begin{center}
    \begin{tabular}{ | l | c |}
      \hline
      CC-BY & Attribution\\
      CC-BY-SA & Attribution + Share Alike\\
      CC-BY-NC & Attribution + Non Commercial\\
      CC-BY-NC-SA & Attribution + Non Commercial + Share Alike\\
      CC-BY-ND & Attribution + No Derivatives\\
      CC-BY-NC-ND & Attribution + Non Commercial + No derivatives\\
      \hline
    \end{tabular}
  \end{center}
  \caption{Available Creative Commons licenses}
  \label{fig:cc_licenses}
\end{table}
\section{Results}
This extension allows participants to \textbf{set a Creative Commons license} to a specific blip, by inserting a gadget in the content and answering questions to reach the adequate license that meets the requirements. The Figure \ref{fig:cc_gadget} shows the final result of this extension.\\[.2cm]
Relating to the gadget structure described in Figure \ref{fig:gadget_classes}, this Gadget's Composite is the class CCGadgetMainPanel, the Messages relates to CCGadgetMessages and the GinModule would be represented by the class CCGadgetGinModule.
\begin{figure}[H]
  \center
    \includegraphics[keepaspectratio, scale=0.7]{Media/Captures/Extensions/CCGadget.png}
  \caption{CCWave}
  \label{fig:cc_gadget}
\end{figure}
This gadgets keeps a variable for each one of the possible concepts for a unique Creative Commons License, such as ``Commercial use'', and \textbf{asks a question to the user} to determine if the restriction has to happen or not, in order to figure out a valid license. The state stores the state of all the answers to the questions that can be asked, and the answers can be yes, no, or unanswered. When a question is answered a state delta is sent to the rest of the participants, and the image of the license is update to the best match from all the possible licenses. When all the answers are answered, only the image of the license remains linking to the whole chosen license.\\[.2cm]
\section{Conclusions and Future Work}
This extension is a Gadget that can be inserted in a blip to let the user select a Creative Commons license for the content of the blip.\\[.2cm]
The main problem with this extension is the \textbf{limited amount of licenses}, only Creative Commons. Even though they are widely used, they might not cater to the tastes of everyone, or fit every possible position. There is also no way to license anything other than a blip, such as a blip and all its children or a whole wave.\\[.2cm]
Those are limitations that can be solved starting with this extension and adding the possibility for more or different licenses. Solving the problem of licenses only being in a blip could not be done with Gadgets as they are by definition inside a blip, it would be compulsory to find a different solution.
\newpage
