\newpage
\section{Technologies}

\begin{itemize}
  \item Wave

  \begin{itemize}

    Wave can be interpreted as different things

    \item Protocol\\
    Google wanted a protocol to replace many aspects of online communication between people (mainly e-mail).\\
    Thus the Google Wave Federation Protocol exists. It is based on Extensible Message and Presence Protocol XMPP, which allows the secure communication of messages based on XML.\\
    The GWFP provides federation on top of XMPP, and it being an open protocol allows anyone to be a wave provider.
    
    \item Wave in a box\\
    Wave can also be used to refer to the software framework that allows users to communicate using GWFP.\\
    Originally called Google Wave, now Apache Wave or Wave in a box WIAB as the server implementation.

    \item Communication unit\\
    Inside WIAB, users are aggregated in what is called a Wave, there they can communicate and create different threads contained in that Wave.\\
    \anotacion{Include diagram showing the Wave(Wavelet(Blip(Document))) structure}    

  \end{itemize}

  \item GWT\\
  Google Web Toolkit GWT is a set of tools that allows web developers to code in Java and from that java-compliant code, generate Asynchronous JavaScript and XML AJAX to make front-end applications. The GWT framework focuses on efficiency and cross-browser compatibility, generating and then serving different AJAX code for every browser and locale to adjust the elements to fit properly.

  \item Gadgets API
  The Gadgets API is an API made by Google to embed third party applications inside various Google products. They implemented it in wave as Wave Gadgets API, to have access to the specific features of Wave. These applications are HTML and JavaScript, with all their possibilities, and access special specific features that the server can offer.

This is a Java API, and it needs to be compiled with the GWT compiler in order to be able to link to the gadgets. It has also a development mode, wich runs native Java and recreates the components without needing a web server.

  \item Robots API
  It is an API with client libraries for Java an Python, both of them with a similar syntax.\\
  The original intention was that robots could act in Wave in exactly the same places and same ways as human participants could. In practice, not all interactions are implemented for robots.\\
  A robot can be added to a Wave, can edit text, add participants, publish new Wavelets, and access the contents of all of them.
\end {itemize}

\begin{itemize}
\item \anotacion{What other things should I talk about?}
\item \anotacion{``Materials'' doesn't make sense, ``Methods'' does. This includes a (technical) explanation/description of the technologies used. You can mention the challenges implied by the lack of documentation, as Google didn't release all the software/docs/APIs/etc. The specific problems go in Results/Discussion, not here, as here the approach is ``before'' implementation.}
  \item Wanted to explore both ways to extend wave
  \item Talk about specific software I used?
  \item How to create a gadget, alternatives, GWT (How is GWT used in wave, what does GWT do)
  \item How to create a robot, alternatives, how much detail?
  \item Servers for gadgets and robots, how to communicate them with wave
  \item Should I talk about the problems I had?
  \item Talk about the lack of documentation?
\end{itemize}
