\newpage
\section{Results}


\begin{itemize}
  \anotacion{Yo no pondría todo esto en ``Results'', sino que haría un capítulo por gadget, antes del capítulo de Results, si te lo permiten (por confirmar). En cada capítulo se puede incluir la misma estructura en pequeño: Introducción (contexto, objetivo), desarrollo (UMLs, etc), resultados/discusión/límites. En cada gadget se puede indicar qué se ha querido testear (teniendo en mente el objetivo de un framework al final). Después de esos capítulos se puede incluir un capítulo de discusión y resultados globales, indicando las lecciones aprendidas y la lista de recomendaciones para el desarrollo de gadgets y robots, en un ``framework'' general que incluya unos UML, enlace a documentación (se puede aprovechar la que Antonio ha generado sobre todo esto en la wiki de Apache Wave, dejando claro que este TFG ha contribuido a su existencia), y un troubleshooting con los problemas encontrados y cómo solucionarlo}
  \item For each project
  \begin{itemize}
    \item Why
    \item how/structure/difficulties
    \item Result
    \item Defects in the result? With how much detail?
  \end{itemize}
  \item Would testing it in real scenarios grant valuable results?
\end{itemize}
