\newpage
\section{Future Work}

There is a lot of work that can be done to Wave, and the wave protocol has a big potential. Apache took control of the project, but the development from their part is slow and very conservative, so there is actually a \textbf{lack of work from part of a big player}. The most work is now being done by the p2pvalue project and Kune, but when pushing changes to the Apache repository, the changes are not always accepted.\\[.2cm]
Wave is in need of features, and some of the ones done here can be considered essential, an they can always be implemented \textbf{closer to the Wave interface and more familiar} regarding the user experience:
\begin{itemize}
  \item Robots: Maybe gadgets and robots are not user-friendly enough. If a user is interested in one of the features a robot can provide, he might not be interested in \textbf{adding an unknown user} to his wave which he wants to keep private.
  \item Gadgets: There is no way of \textbf{seeing all of the gadgets that are available}. If a gadget is not added to the list of gadgets, you need to know the exact URL of the gadget to use it, that means having to know the gadget beforehand. There are, too, many options when trying to choose a gadget, and you need to test them to see if they are the ones needed.
\end{itemize}
When developing the Decision Maker Gadget, because of its slightly complex state storing structure, it was clear that the way of storing state as key-value pairs is too limiting. Storing complex values as strings means you have to create a special format depending on the nature of the object to store, storing lists means having to use a character as a separator of the elements rendering that character unable to be used inside the elements themselves. A good solution for this would be to make a \textbf{serializer library} that generates XML or JSON documents representing the structure and value of objects, and submits and retrieves values from the  Wave's state in an easy way.

\addcontentsline{toc}{section}{\numberline{}Bibliography}

