\chapter{Concluding Remarks}
The whole end-of-degree project is analysed in the following sections regarding the results of the work, the conclusion reached after its development, and future lines of work that could be followed after making this work.
\section{Discussion of Overall Results}
%\addcontentsline{toc}{chapter}{\numberline{Global Results}}
Developing for Wave is \textbf{full of difficulties}. When Google released the project to the Apache Foundation to put into the Incubator they didn't fully release everything, keeping parts of the software hidden. There was lack of documentation for many features, and over time links with important information about Wave have been deleted without replacement. This informality leaves the developer with the only option of trial and error to make things work.\\[.2cm]
Many of these extensions, specifically the Creative Commons License Gadget, the Colorizer Robot and the Video Conference Gadget, or better said the base concepts behind them, had already been suggested for the P2Pvalue project to build, so they meet actual needs that the Wave community has right now.\\[.2cm]
There are attempts to \textbf{increase the accessibility to Wave} not only to users, but also to developers. That means improving visibility of all the projects related to Wave, completing the features that Wave is lacking right now, and documenting important things that act as tall walls for a developer that wants to get into it. This end-of-degree project contributed to review documentation about Wave gadgets and robots and add to it, as well as updating the sample gadget provided in the Kune repository to one working for the latest version of Kune and a newer version of the GWT.\\[.2cm]
It would have been useful to make a \textbf{bigger framework}, with more variety and amount of gadgets. Suggested future work in the previous chapters is sometimes outside of the purpose of a framework for gadgets and robots, but the application framework itself can also be improved: Because of the increased difficulty in getting into the development there are procedures that were intended to be documented but haven't been. Also, this work is only as useful as the people who see it, then it will benefit from being spread.\\[.2cm]
The reasons for developing each of the extensions, as well as the contributions to documentation have been explained in the chapter for each extension, but Table \ref{fig:contributions} shows them summarised.
\begin{table}[h]
  \footnotesize
  \begin{center}
    \begin{tabular}{ | l | p{2.49cm} | p{2.1cm} | p{2.2cm} | p{2.3cm} |}
      \hline
      \textbf{Extension} & \textbf{Need Satisfied} & \textbf{Features} & \textbf{Interacts With} & \textbf{Contributions}\\
      \hline
      \ref{subsec:cc_intro} - CCWave & D1.3 Design Guidelines (Section Legal Regime) \cite{ref:p2pvalue} & Choose a Creative Commons License for the content of a blip & Gadgets API & Gadget Development Tutorial \cite{ref:gadget_development}, CCWave \cite{ref:cc_github}\\
      \hline
      \ref{subsec:decision_intro} - Pollymer & D1.3 Design Guidelines (Section Technical Features) \cite{ref:p2pvalue}) & Poll users about a question & Gadgets API & Gadget Development Tutorial \cite{ref:gadget_development}, Pollymer \cite{ref:decision_github}\\
      \hline
      \ref{subsec:video_intro} - AppearWOW & D1.3 Design Guidelines (Section Technical Features) \cite{ref:p2pvalue} & Video conference communication & Gadgets API, appear.in & AppearWOW \cite{ref:video_github}\\
      \hline
      \ref{subsec:color_intro} - Colbotia & D1.3 Design Guidelines (Section Economic Model) \cite{ref:p2pvalue} & Highlight the contributions of each participant & Robots API, OAuth & Development Tutorial \cite{ref:robot_development}, Sample README \cite{ref:readme_sample}, Registration Process \cite{ref:registration_process}, Colbotia \cite{ref:colorizer_github}\\
      \hline
    \end{tabular}
  \end{center}
  \caption{Extensions Summary}
  \label{fig:contributions}
\end{table}
\section{Conclusion}
%\addcontentsline{toc}{chapter}{\numberline{Conclusion}}
Gadgets by nature are just embedded interactuable applications, slightly decoupled from the context they are in. With increasing computing power, even in mobile devices, the big difference \cite{ref:javascript_slow} in power needed to complete a task between JavaScript and other native languages starts getting negligible. GWT achieves \textbf{high compatibility} with all different browsers, and with the help of different APIs can be a very good tool for creating gadgets in situations where the concept of the gadget makes sense. The only limitation is it is not possible to get outside of the gadget apart from what the API provides.\\[.2cm]
Gadgets are a relatively \textbf{easy way to extend Wave}, as no actual knowledge of Wave's codebase is needed to do so. But still the Gadgets API is far from perfect, and requires the developer to repeat tasks for every gadget made. Also, there will be sometimes version compatibility problems, project configuration problems, classpath errors, and the messages given by the GWT compiler are not informative enough. Many steps need to be followed to get things working.\\[.2cm]
Also, even though the Gadgets API is a Java API, the fact that it generates JavaScript code means troubleshooting is sometimes needed in the JavaScript layer, so JavaScript knowledge is also needed, and computer generated code can sometimes not be easy to follow and understand.\\[.2cm]
The project aims to take advantage of the features in Wave and solve the issues and difficulties that could arise for developers when programming Gadgets and Robots. To achieve it, a framework for development and complete real extensions have been developed.\\[.2cm]
As a framework, successful development of both gadgets and robots has been carried out completely from start to finish, using \textbf{all the important and interesting features} provided by the Gadgets API and Robots API. Not only that, but also issues have been documented and samples have been provided to the public to use.\\[.2cm]
Useful gadgets have also been developed. They were inspired by \textbf{actual needs}, and have been finished to a ready to use state. Anyway, they are openly available under the Affero General Public License\cite{ref:agpl} for anyone to use, modify, and continue improving them.

\section{Future Work}
%\addcontentsline{toc}{chapter}{\numberline{Future Work}}
There is a lot of work that can be done to Wave, and the wave protocol has a big potential. Apache took control of the project, but the development from their part is slow and very conservative, so there is actually a \textbf{lack of work from part of a big player}. The most work is now being done by the P2Pvalue project and Kune, but when pushing changes to the Apache repository, the changes are not always accepted.\\[.2cm]
Wave is in need of features, and some of the ones done here can be considered essential, an they can always be implemented \textbf{closer to the Wave interface and more familiar} regarding the user experience:
\begin{itemize}
  \item Robots: Maybe gadgets and robots are not user-friendly enough. If a user is interested in one of the features a robot can provide, he might not be interested in \textbf{adding an unknown user} to his wave which he wants to keep private.
  \item Gadgets: There is no way of \textbf{seeing all of the gadgets that are available}. If a gadget is not added to the list of gadgets, you need to know the exact URL of the gadget to use it, that means having to know the gadget beforehand. There are, too, many options when trying to choose a gadget, and you need to test them to see if they are the ones needed.
\end{itemize}
When developing the Decision Maker Gadget, because of its slightly complex state storing structure, it was clear that the way of storing state as key-value pairs is too limiting. Storing complex values as strings means you have to create a special format depending on the nature of the object to store, storing lists means having to use a character as a separator of the elements rendering that character unable to be used inside the elements themselves. A good solution for this would be to make a \textbf{steriliser library} that generates XML or JSON documents representing the structure and value of objects, and submits and retrieves values from the  Wave's state in an easy way.

\addcontentsline{toc}{chapter}{\numberline{}Bibliography}

\rhead{}
\renewcommand{\headrulewidth}{0pt}
