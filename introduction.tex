\newpage
\thispagestyle{sectioned}
\pagenumbering{arabic}
\chapter{Introduction}
%\addcontentsline{toc}{chapter}{\thechapter{} Introduction}
Nowadays online social networks such as Facebook and Google+ are a big part of the mainstream Internet usage. There have been countless social networks in a very short span of time, and each time they are different and innovative, and are growing importance offline.\\[.2cm]
One of those remarkable attempts was \textbf{Google Wave}, that had a short life before Google dropped support and development, and even took the working nodes down. However the control was given to Apache to continue development of it. Since the inception of Wave, the web has undergone several changes, but the technology and protocol still remain relevant. JavaScript is one of the most used resources to execute code on the client-side\\[.2cm].
The world has also faced towards the cloud: applications are increasingly extracted from the personal computer and stored in a remote location to interact and share. One of the biggest advantages of the cloud is that it facilitates sharing, distributing creations among people. There is a similar concept to the one of sharing: \textbf{collaboration}, or being able to use the software simultaneously and in real time (Or almost real time) with other people and seeing the results. Wave embraced closely the concept of collaboration, and everything in it was geared towards it. Another successful Google collaborative product is Google Docs \cite{ref:google_docs} that predates Wave, as a consequence since then there have been other collaborative applications like Zoho \cite{ref:zoho}, Pads (Which are explained in Section \ref{subsec:color_soa}) and more.\\[.2cm]
With the growth of Facebook they began increasing the variety of features, and one of them that relates to this work is the arrival of Facebook Apps: applications developed by third parties but integrated inside Facebook by using some of its features.\\[.2cm]
When Google announced Wave it was an innovative product, and attracted positive attention. Google, disappointed with the adoption Wave received, and despite the enthusiasm from developers, stopped the development of Wave. They did not completely make it disappear, and instead chose to hand the project over to Apache to continue development, who deposited it on the Apache Incubator, a program for supporting promising open source software with the objective of converting it in an Apache Foundation project once it is mature enough.\\[.2cm]
Two of the important features of Wave since the beginning have been Robots and Gadgets: ways to \textbf{extend the capabilities of Wave} without the need to integrate them in the actual software, they can be made and changed independently. Gadgets are separate apps embedded inside another more powerful entity that hosts them. Robots interact with that same entity and are able to react to changes in it. The overall objective of this work is to create a framework around Wave, based on Gadgets and Robots, to explore their functionalities, help document their features and usage, and try to build a base so this two extensions can continue being explored and expanded in the future.\\[.2cm]
The P2Pvalue \cite{ref:p2pvalue} project has inspired the existence of this work and what is needed to be done. The P2Pvalue is a ``Techno-social platform for sustainable models and value generation in commons-based peer production in the Future Internet'', and are studying different aspects of social networks, and building an online collaboration platform on top of Wave to promote ``communities of collaborative production''.\\[.2cm]
The aim of this work is a framework, understood as the structure supporting a conglomerate of elements with a similar purpose or context. When Google dropped support for Wave, they also began sparingly removing web pages with very important information about it, and some of that information has been lost.

\section{Objectives}
This end-of-degree project was not intended to be just another exercise, but rather accomplish actual contributions to a real existing project. It then tries to put together every piece regarding gadgets and robots, test everything relating to them and lay a rough path so it can be followed in the future. This involves \textbf{documenting and testing} all the features provided by the available APIs.\\[.2cm]
However that is not the only intention, it was also important to make some tangible \textbf{extensions as robots and gadgets} that are useful for the Wave community, that satisfy needs not yet met. Everything has been open sourced and released under a free license \cite{ref:agpl} so it can be further improved and analyzed. The project P2Pvalue \cite{ref:p2pvalue} has been a great influence on what kind of features are needed, and some of them have been successfully accomplished.
\begin{itemize}
  \item {
    Wave Documentation
    \begin{itemize}
      \item Improve existing documentation
      \item Document undocumented features
    \end{itemize}
  }
  \item {
    Projects
    \begin{itemize}
      \item Updating existing projects
      \item Generating new useful extensions
    \end{itemize}
  }
  \item {
    Usage
    \begin{itemize}
      \item Explaining undocumented procedures
      \item Teaching how to start development of Wave extensions
    \end{itemize}
  }
\end{itemize}

\section{Document Structure}
All of those elements that compound the framework are explored individually. Those elements can be described individually, but the framework as a whole is also described. There are \textbf{global results and individual results}, and they are all organized as follows: first there is a global State of the Art and Methods, then each element is completely explained individually with its introduction, state of the art, results and conclusions, and finally global conclusions are reached. Also, every extension have their own results, conclusions and future work, but the complete work is analyzed as a single entity.\\[.2cm]
The objectives of the framework itself are the ones stated above, but every extension is also a whole project by themselves, so they have their specific objectives.
\begin{itemize}
  \item Chapter 3 - State of the Art: alternatives to Wave gadgets and robots.
  \item Chapter 4 - Technologies and methods: every procedure and technology that has been used along this work.
  \item Chapter 5 - Frame for Development of Gadgets \& Robots: how Gadgets and Robots work. How to get them running.
  \item Chapter 6 - CCWave: development of a Wave gadget for choosing a Creative Commons License.
  \item Chapter 7 - Pollymer: development of a Wave gadget for decision making.
  \item Chapter 8 - AppearWOW: development of a Wave gadget for video conference.
  \item Chapter 9 - Colbotia: development of a Wave robot for colorizing the contributions of each participant.
  \item Chapter 10 - Concluding Remarks: comments and analysis on the whole project.
\end{itemize}

\newpage
\thispagestyle{sectioned}
\chapter{Introducción}
%\addcontentsline{toc}{chapter}{\thechapter{} Introduction}
Hoy en día las redes sociales en línea tales como Facebook o Google+ son de gran importancia en el uso convencional de Internet. Numerosas redes sociales han surgido en un corto período de tiempo, han sido diferentes e innovativas, y disfrutan de una creciente importancia fuera de la red.
Uno de esos destacables intentos fue \textbf{Google Wave}, que tuvo una breve existencia antes de que Google decidiera dejar de dar soporte para él y desarrollarlo, e incluso desactivó los nodos que habían estado funcionando hasta entonces. El control del proyecto le fue otorgado a Apache para que continuara con su desarrollo. Desde el origen de Wave, la web ha sufrido numerosos cambios, pero la tecnología y el protocolo de Wave siguen hoy en día siendo relevantes. JavaScript es una de las tecnologías web más usadas hoy en día para ejecutar código en el cliente.
El mundo ha girado en el sentido de ``la nube'': las aplicaciones cada vez más son extraídas del ordenador del usuario y almacenadas en una localización remota para interactuar con ellas y compartirlas. Una de las principales ventajas de ``la nube'' es que facilita el intercambio entre usuarios, distribuir creaciones entre otras personas. Existe un concepto similar al de intercambio: \textbf{colaboración}, o usar aplicaciones de manera simultánea con otras personas y en tiempo real (O tiempo casi real) para ver el resultado a la vez. Wave siempre mantuvo cerca el concepto de colaboración, y todo en él estaba diseñado para colaborar. Otro exitoso producto de Google relacionado con la colaboración es Google Docs \cite{ref:google_docs}, que antedata a Wave, y por ellos desde entonces han existido otras aplicaciones colaborativas como Zoho \cite{ref:zoho}, Pads (Explicados en la Sección \ref{subsec:color_soa}) y otros.\\[.2cm]
Acompañando al crecimiento de la red social Facebook, desarrollaron varias nuevas utilidades, y una de ellas está muy relacionada con este trabajo y son las Apps Facebook: aplicaciones desarrolladas por terceros e integradas dentro de Facebook usando algunas de sus características.\\[.2cm]
Cuando Google anunció Wave fue una innovación, y atrajo una atención positiva. Google, decepcionados por una mala acogida del producto, a pesar de la buena aceptación por parte de los desarrolladores, decidieron ponerle fin al proyecto. No hicieron que desapareciera completamente, en su lugar eligieron otorgarle el proyecto a Apache con la intención de que continuara el desarrollo, y lo pusieron en el Apache Incubator, un programa para ayudar a prometedores proyectos de software libre con el objetivo de acabar incorporándolos como proyectos de la Apache Foundation cuando estén lo suficientemente maduros.\\[.2cm]
Dos de las características de Wave desde el principio han sido los Robots y los Gadgets: formas de \textbf{extender las capacidades de Wave} sin la necesidad de integrar nada en el software de Wave, así que pueden ser hechos y modificados de manera independiente. Los Gadgets son aplicaciones separadas pero empotradas dentro de una entidad más potente que ellos y que se encarga de contenerlos. Los Robots interactúan con esa misma entidad y pueden reacciónar a sus cambios y eventos. El objetivo general de este trabajo es crear un marco alrededor de Wave basado en Gadgets y Robots y así explorar sus funcionalidades, ayudar a documentar sus características y uso, además de cimientar una base para que estos dos tipos de extensiones puedan seguir siendo exploradas y mejoradas en un futuro.\\[.2cm]
El proyecto P2Pvalue \cite{ref:p2pvalue} ha inspirado tanto la existencia de este trabajo así como lo que era necesario hacer. P2Pvalue es una ``Plataforma tecno-social para modelos sostenibles y generación de valor en la producción de iguales basada en lo común en el Internet Futuro'', están estudiando diferentes aspectos de las redes sociales, y construyendo una plataforma online colaborativa basada en Wave para promover ``comunidades de producción colaborativa''.\\[.2cm]
El objectivo de este proyecto es un framework, entendido como la estructura que soporta un conglomerado de elementos con una intención o contexto similares. Cuando Google dejó de dar soporte para Wave, también empezaron lentamente a eliminar páginas web con información muy importante sobre Wave, y partes de esa información se han perdido.

\section{Objetivos}
Para este proyecto de fin de grado estaba pensado que no fuera simplemente un ejercicio más, sino llevar a cabo contribuciones reales a proyectos existentes. Por lo tanto se unen todas las piezas relacionadas con los Gadgets y Robots, se  prueba todo lo que tenga que ver con ellos y se dibuja un camino que podrá ser seguido más adelante. Para ello hay que \textbf{documentar y probar} todas las características que se ponen a nuestra disposición gracias a todas las APIs.\\[.2cm]
De todas maneras ésa no es la única intención, también es importante hacer algunas aportaciones tangibles en forma de \textbf{extensiones como Robots y Gadgets} que sean útiles para la comunidad de Wave, y que satisfagan necesidadas existentes que todavía no han sido solventadas. Todo el proceso se ha liberado en código y otorgado una licencia libre \cite{ref:agpl} para que otras personas puedan analizarlo y mejorarlo. El proyecto P2Pvalue \cite{ref:p2pvalue} ha sido una gran influencia en cuanto a qué características eran más necesitadas, y algunas de ellas han sido completadas satisfactoriamente.
\begin{itemize}
  \item {
    Documentación de Wave
    \begin{itemize}
      \item Mejorar documentación existente
      \item Documentar características no documentadas
    \end{itemize}
  }
  \item {
    Proyectos
    \begin{itemize}
      \item Actualizar proyectos ya existentes
      \item Generar nuevas extensiones útiles
    \end{itemize}
  }
  \item {
    Uso
    \begin{itemize}
      \item Explicar procedimientos no documentados
      \item Enseñar cómo introducirse al desarrollo de extensiones Wave
    \end{itemize}
  }
\end{itemize}

\section{Estructura del Documento}
Todos esos elementos que componen el framework serán explorados de manera individual. Pueden ser descritos individualmente, pero a su vez se puede describir el framework como unidad. Hay disponibles \textbf{resultados globales así como individuales}, y todos ellos están organizados de la siguiente manera: en primer lugar se encuentra un Estado del Arte general así como unos Métodos, y después cada elemento se explica completamente de manera indificual con su propia introducción, estado del arte, resultados y conclusiones. Finalmente se alcanzan unas conclusiones globales. Además de eso, cada extensión tiene sus propios resultados, conclusiones y trabajo futuro, pero el trabajo completo es analizado también como una entidad individual.\\[.2cm]
Los objetivos del framework en sí mismo son los descritos arriba, pero todas las extensiones son un proyecto completo en ellas mismas, por lo que tienen objetivos específicos.
\begin{itemize}
  \item Capítulo 3 - Estado del Arte: alternativas a los Robots y Gadgets de Wave.
  \item Capítulo 4 - Métodos y Tecnologías: cada procedimiento y tecnología que hayan sido usados a lo largo de este trabajo.
  \item Capítulo 5 - Marco para el Desarrollo de Gadgets y Robots: cómo funcionan los Gadgets y Robots. Cómo se pueden ejecutar.
  \item Capítulo 6 - CCWave: desarrollo de un gadget de Wave para elegir una licencia Creative Commons.
  \item Capítulo 7 - Pollymer: desarrollo de un gadget de Wave para toma de decisiones.
  \item Capítulo 8 - AppearWOW: desarrollo de un gadget de Wave para realizar vídeo conferencias.
  \item Capítulo 9 - Colbotia: desarrollo de un robot de Wave para colorear las contribuciones de los participantes.
  \item Capítulo 10 - Comentarios Conclusivos: comentarios y análisis del proyecto completo.
\end{itemize}
