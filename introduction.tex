\newpage
\pagenumbering{arabic}

\section{Introduction}
Nowadays social networks are a big part of the mainstream Internet usage. There have been countless social networks in a very short span of time, and each time they are different and innovative, and are important even in actual society outside of the internet.\\[.2cm]
One of those remarkable attemps was \textbf{Google Wave}, that had a short life before Google dropped support and development, and even took the working nodes down. But the control was given to Apache to continue development of it. Since the inception of Wave, the web has undergone several changes, but it still keeps being relevant. JavaScript is one of the most used resources to execute code on the client-side, and Wave is built on top of JavaScript.\\[.2cm]
The world has also faced towards the cloud: Applications are more and more each day taken out of the personal computer and put in a remote location to store and share. One of the biggest advantages of the cloud is the sharing: Distributing creations among real people. The extreme of this \textbf{social sharing is collaboration}, or being able to use the software simultaneously and in real time (Or almost real time) with other people and seeing the results. Wave embraced closely the concept of collaboration, and everything in it was geared towards it. Since then, there has been other collaborative applications like Google Docs \cite{ref:google_docs}, Zoho \cite{ref:zoho}, Pads and more.\\[.2cm]
Two of the important features of Wave since the beginning have been Robots and Gadgets: Ways to \textbf{extend the capabilities of Wave} without the need to integrate them in the actual software, they can be made and changed individually. Gadgets are separate apps embedded inside another more powerful entity that hosts them. Robots interact with that same entity and are able to react to changes in it. The overall objective of this work is to create a framework around Wave, based on Gadgets and Robots, to explore their functionalities, help document their features and usage, and try to build a base so this two extensions can continue being explored and expanded in the future.

\subsection{Objectives}
The aim of this work is a framework, understood as the structure supporting a conglomerate of elements with a similar purpose or context. When Google dropped support for Wave, they also began sparingly removing web pages with very important information about it, and some of that information has been lost. This work tries to put together every piece regarding gadgets and robots, test everything relating to them and lay a rough path so it can be followed in the future. This involves \textbf{documenting and testing} all the features provided by the available APIs.\\[.2cm]
But that is not the only intention, it was also important to make some tangible \textbf{extensions as robots and gadgets} that are useful for the Wave community, that satisfy needs not yet met. Everything has been open sourced and released under a free license \cite{ref:agpl} so it can be further improved and analyzed. The project P2Pvalue \cite{ref:p2pvalue} has been a great influence on what kind of features are needed, and some of them have been succesfully accomplished.

\subsection{Document Structure}
All of those elements that compound the framework are explored individually. Those elements can be described individually, but the framework as a whole is also described. There are global results and individual results, and they are all organized as follows: First there is a global State of the Art and Methods, then each element is completely explained individually with its introduction, state of the art, results and conclusions, and finally global conclusions are reached.\\[.2cm]
The objectives of the framework itself are the ones stated above, but every extension is also a work by themselves, so they have their specific objectives.
