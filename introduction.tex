\newpage
\pagenumbering{arabic}
\section{Introduction}
Nowadays social networks are a big part of the mainstream Internet usage. There has been countless social networks in a very short span of time, and each time they are different and innovative, and are important even in actual society outside of the internet.\\[.2cm]
One of those remarkable attemps was Google Wave, that had a short life before Google dropped support and development, and even took the working nodes down. But the control was given to Apache to continue development of it. Since the inception of Wave, the web has undergone several changes, but it still keeps being relevant. JavaScript is one of the most used resources to execute code on the client-side, and Wave is built on top of JavaScript.\\[.2cm]
The world has also faced towards the cloud: Applications are more and more each day taken out of the personal computer and put in a remote location to store and share. One of the biggest advantages of the could is the sharing: Distributing creations among real people. The extreme of this social sharing is collaboration, or being able to use the software simultaneously and in real time (Or almost real time) with other people and seeing the results. Wave embraced closely the concept of collaboration, and everything in it was geared towards it. Since then, there has been other collaborative applications like Google Docs, Zoho, Pads...\\[.2cm]
Two of the important features of Wave have been since the beginning Robots and Gadgets: Ways to extend the capabilities of Wave without the need to integrate them in the actual software, they can be made and changed individually. The aim of this work is to create a framework around Wave, based on Gadgets and Robots, to explore their functionalities, help document their features and usage, and try to build a base so this two extensions can continue being explored and expanded in the future.
