\newpage
\thispagestyle{sectioned}
\section{Conclusion}

Gadgets by nature are just embedded interactuable applications, slightly decoupled from the context they are in. With increasing computing power, even in mobile devices, the big difference \cite{ref:javascript_slow} in power needed to complete a task between JavaScript and other native languages starts getting neglegible. GWT achieves \textbf{high compatibility} with all different browsers, and with the help of different APIs can be a very good tool for creating gadgets in situations where the concept of the gadget makes sense. The only limitation is it is not possible to get outside of the gadget apart from what the API provides.\\[.2cm]
Gadgets are a relatively \textbf{easy way to extend Wave}, as no actual knowledge of Wave's codebase is needed to do so. But still the Gadgets API is far from perfect, and requires the developer to repeat tasks for every gadget made. Also, there will be sometimes version compatibility problems, project configuration problems, classpath errors, and the messages given by the GWT compiler are not informative enough. Many steps need to be followed to get things working.\\[.2cm]
Also, even though the Gadgets API is a Java API, the fact that it generates JavaScript code means troubleshooting is sometimes needed in the JavaScript layer, so JavaScript knowledge is also needed, and computer generated code can sometimes not be easy to follow and understand.\\[.2cm]
The wishes by which this work was inspired have definitely been satisfied.\\[.2cm]
As a framework, succesful development of both gadgets and robots has been carried out completely from start to finish, using \textbf{all the important and interesting features} provided by the Gadgets API and Robots API. Not only that, but also issues have been documented and samples have been provided to the public to use.\\[.2cm]
Useful gadgets have also been developed. They were inspired by \textbf{actual needs}, and have been finished to a ready to use state. Anyway, they are openly available under the Affero General Public License\cite{ref:agpl} for anyone to use, modify, and continue improving them.

