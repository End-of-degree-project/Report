\newpage
\renewcommand{\thepage}{\Roman{page}}
\setcounter{page}{9}
\chapter*{Abstract}
\addcontentsline{toc}{chapter}{\numberline{}Abstract}
Real-time online collaborative software grants the user a broad collection of alternatives including, yet not limited to, concurrent document edition, gaming and technical applications. It is a field of increasing importance and still not fully exploited potential.\\[.2cm]
Apache Wave provides a standard for federated real-time collaboration allowing the usage of small applications with collaborative purpose named Gadgets, embedded in particular online documents. However the amount of activity regarding Apache Wave declined in the recent past and there exists a concerning lack of information relative to the development of Gadgets. Accompanying that, there are also important needs solvable with Gadgets that have not been implemented.\\[.2cm]
This work explores the scene of Wave, and specifically attemps to set the base for real-world developers to code collaborative applications under the Wave infrastructure. In the frame of this work Several different extensions to Apache Wave have been developed taking advantage of open source software, open protocols and open standards. Besides, undocumented features and procedures to help developers have been detailed.\\[.2cm]
The developed extensions are the following: a gadget to set an open license to your content, a gadget for decision making, a gadget for video conference, and another extension for highlighting the contributions of each participant in a document. Information about the development and use of extensions has been documented and shared publicly.\\[.2cm]
The results of this project constitute a framework that aids developers attemping to code extensions for Apache Wave by gathering and creating information about the process, as well as implementing four extensions that are both needed in the community and representative of key functionalities.
\vfill
{\large \bf Keywords:}\\
{\large Apache Wave, Collaboration, Gadgets, Apps, Real Time, Federation, Google Web Toolkit}

\newpage
\renewcommand{\thepage}{\Roman{page}}
\setcounter{page}{10}
\chapter*{Resumen}
\addcontentsline{toc}{chapter}{\numberline{}Resumen}
El software colaborativo en tiempo real le otorga al usuario una amplia variedad de alternativas, entre ellas edición concurrente de documentos, vídeo juegos y aplicaciones técnicas. Es un campo de creciente importancia y todavía no explotado completamente.\\[.2cm]
Apache Wave facilita un estándar para colaboración federada en tiempo real permitiendo el uso de pequeñas aplicaciones colaborativas llamadas Gadgets, integradas en ciertos documentos online. La actividad en cuanto a Apache Wave ha ido disminuyendo recientemente y existe una preocupante falta de información relativa al desarrollo de Gadgets. Asimismo, también hay importantes necesidades solventables con Gadgets que todavía no han sido resueltas.\\[.2cm]
Este trabajo explora el panorama de Wave, y concretamente intenta establecer la base para que los desarolladores del mundo real puedan programar aplicaciones colaborativas bajo la infraestructura Wave. En el marco de este trabajo diversas extensiones de Apache Wave han sido desarolladas haciendo uso de software libre, así como protocolos y estándares libres. Asimismo, características previamente no documentadas y procedimientos para ayudar a los desarrolladores han sido detallados.\\[.2cm]
Las extensiones desarrolladas son las siguientes: un Gadget para asignar una licencia libre a tu propio contenido, un Gadget de toma de decisiones, un Gadget para videoconferencia, y otra extensión para colorear las contribuciones de cada participante en un documento. Se ha documentado y compartido públicamente información sobre el desarrollo y uso de las extensiones.\\[.2cm]
El resultado de este proyecto genera un framework el cual facilita a los desarrolladores pretendiendo programar extensiones de Apache Wave recolectando y creando información acerca del proceso, además de implementar extensiones necesarias en la comunidad y representativas de funcionalidades clave.
\vfill
{\large \bf Palabras Clave:}\\
{\large Apache Wave, Colaboración, Gadgets, Apps, Tiempo Real, Federación, Google Web Toolkit}

\newpage
\thispagestyle{empty}
\mbox{}

