\newpage
\renewcommand{\thepage}{\Roman{page}}
\setcounter{page}{9}
\chapter*{Abstract}
\addcontentsline{toc}{chapter}{\numberline{}Abstract}
Real-time online collaborative software grants the user a broad collection of alternatives including, yet not limited to, concurrent document edition, gaming, and technical applications. It is a field of increasing importance and still not fully exploited potential.\\[.2cm]
Apache Wave provided an standard for federated real-time collaboration allowing the usage of small applications with collaborative purpose called Gadgets embedded in particular online documents. However the amount of activity regarding Apache Wave declined in the recent past and there exists a concerning lack of information relative to the development of Gadgets. Accompanying that, there are also important needs solvable with Gadgets that have not been executed.\\[.2cm]
This work explores the scene of Wave, and specifically attemps to set the base for developers to code collaborative applications under the Wave infrastructure. Several different extensions to Apache Wave have been developed taking advantage of open software, protocols and standards. Undocumented features and procedures to help developers have been detailed.\\[.2cm]
The developed extensions are the following: a Gadget to set an open license to your content, a Gadget for decision making, a Gadget for video conference, and another extension for highlighting the contributions of each participant in a document. Information about the development and use of extensions has been documented and shared publicly.\\[.2cm]
There is not a frame for developers to gather information on how to program extensions for Apache Wave, this work fills that gap between developers and Apache Wave by solving different weaknesses that previously existed.
\vfill
{\large \bf Keywords:}\\
{\large Apache Wave, Collaboration, Gadgets, Apps, Real Time, Federation, Google Web Toolkit}

\newpage
\renewcommand{\thepage}{\Roman{page}}
\setcounter{page}{10}
\chapter*{Resumen}
\addcontentsline{toc}{chapter}{\numberline{}Resumen}

\vfill
{\large \bf Palabras Clave:}\\
{\large Apache Wave, Colaboración, Gadgets, Apps, Tiempo Real, Federación, Google Web Toolkit}

\newpage
\thispagestyle{empty}
\mbox{}

