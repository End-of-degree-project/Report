
\newpage
\section{State of the art}
\ant{Empieza todas las secciones con una introducción. Continua desarrollando las ideas organizadas temáticamente. Finaliza resumiendo lo que has presentado en la sección e introduciendo la sección siguiente}

\ant{En el state of the art, presenta los trabajos relacionados, indicando las diferencias con tu propuesta.}

\begin{itemize}
  \item Apache Wave Project
  \begin{itemize}
    \item link: incubator.apache.org/wave/
    \item The Apache Wave Project is an incubated project which aims for flexible and dynamic ways of comunication. The main focus of this project is the development of Wave in a Box WiaB, that is a continuation started when Google stopped the development of Google Wave, and tries to complete what Wave was meant to be. The project works towards making reusable components that can be used in other Wave-related products. The project is yet away from reaching a state other than incubation.
  \end{itemize}
  \item Kune
  \begin{itemize}
    \item \anotacion{Am I lying? Where can I find sources for this?}
    \item Kune was originally an independent software, later forked Google Wave to take advantage of its features. Kune keeps now the main Apache Wave's characteristics, but adds a wrapper around giving it a personal space to each user, creative commons licenses to the contents, amongh other extensions.
    \item Kune's main intention is to promote real-time collaboration, as oposed to standard communication, even bidirectional.
  \end{itemize}
  \item Gadgets and robots (Not only wave's)
    \begin{itemize}
      \item Gadgets and robots are present in multiple places of the web space.
      \item Embedded applications
      \begin{itemize}
        \item There are multiple different ways to embed applications or elements in different places of a webpage. In the case of Wave, Google decided to call those elements Gadgets, and Gadgets have outlived Google's Wave and keep being used with the Gadgets API.
        \item The nature of the web nowadays is very reliant on JavaScript [link], and JavaScript makes it easy to embed fully featured applications almost anywhere in a modern browser, as most of them support it [link to support of JavaScript support on different browsers]. But there's other alternatives: Java Applets, Flash components... Wave's Gadgets are JavaScript components inside an iframe.
      \end{itemize}
      \item Robots
      \begin{itemize}
        \item Robots are components on the internet whose objective is to parse content, mainly text [link], and react to it. They can generate reports, gather mass information, notify of events, or even act indistinnguishibly from a common user.
        \item Cons about robots, captchas, robots.txt...
        \item Robots in Kune, they try to avoid the negative aspects of robots
      \end{itemize}
    \end{itemize}
  \item Known GWT projects
  \begin{itemize}
    \item Whirled - http://www.whirled.com/\#world-places
    \item https://www.scenechronize.com/tvshows.php
    \item http://www.gogrid.com/
    \item http://www.google.com/moderator/
  \end{itemize}
  \item Other social networks (facebook...)
  \begin{itemize}
    \item Facebook: pictures, tags, comments, focused on profiles...
    \item Google+: Similar to facebook, trying not to be so personal
    \item No successfull open source alternatives, mention this?
    \item More?
  \end{itemize}
  \item Other collaborative tools (Google Docs, Zoho...)
  \begin{itemize}
    \item Zoho
    \begin{itemize}
      \item Focused on enterprise
      \item More?
    \end{itemize}
    \item Google Docs
    \begin{itemize}
      \item Classical office, but with collaboration
      \item More?
    \end{itemize}
  \item \ant{PADs, e.g., PiratePad (de donde recoges la idea de colorear las intervenciones de cada usuario)} 
  \end{itemize}
  
\end{itemize}
